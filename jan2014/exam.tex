\documentclass[danish,a4paper]{report}
\usepackage[T1]{fontenc}
\usepackage[utf8]{inputenc}
\usepackage{lmodern}
\usepackage{hyperref}
\usepackage{graphicx}
\usepackage[english]{babel}
\usepackage{verbatim}
\usepackage{color}
\usepackage[table]{xcolor}
\usepackage{graphicx}
\usepackage{parskip}
\usepackage{url}
\usepackage{background}
\usepackage{lastpage}
\usepackage{titlesec}
\usepackage{blindtext}
\usepackage{color}
\usepackage{nopageno}
\usepackage{todonotes}
\usepackage{listings}


\definecolor{dkgreen}{rgb}{0,0.6,0}
\definecolor{gray}{rgb}{0.5,0.5,0.5}
\definecolor{mauve}{rgb}{0.58,0,0.82}
\definecolor{gray75}{gray}{0.75}
\definecolor{light-gray}{gray}{0.5}

\lstset{
  frame=,
  language=C,
  aboveskip=3mm,
  belowskip=3mm,
  showstringspaces=false,
  columns=flexible,
  basicstyle={\small\ttfamily},
  numbers=none,
  numberstyle=\tiny\color{gray},
  keywordstyle=\color{blue},
  commentstyle=\color{dkgreen},
  stringstyle=\color{mauve},
  breaklines=true,
  breakatwhitespace=true
  tabsize=4
}

\addtolength{\topmargin}{-1.5cm}
\addtolength{\textheight}{1.5cm}
\addtolength{\oddsidemargin}{-1cm}
\addtolength{\evensidemargin}{-1cm}
\addtolength{\textwidth}{2cm}

 
\definecolor{schultz}{RGB}{146,34,126}
\definecolor{gray75}{gray}{0.75}
\definecolor{light-gray}{gray}{0.5}
\newcommand{\hsp}{\hspace{20pt}}
\titleformat{\chapter}[hang]{\Huge\bfseries}{\thechapter\hsp}{0pt}{\Huge\bfseries}[{\titlerule[1.5pt]}]
% \titlespacing*{\chapter}{0cm}{-10pt}{2em}
\titlespacing*{\chapter}{0cm}{-20pt}{2em}
\setcounter{secnumdepth}{3}
\setcounter{tocdepth}{1}

\titlespacing*{\section} {0pt}{5pt}{2pt}
\titlespacing*{\subsubsection}{0pt}{0pt}{0pt}

\backgroundsetup{
  scale=1,
  color=black,
  opacity=1,
  angle=0,
  position=current page.south,
  vshift=60pt,
  contents={%
  \small%
  \begin{minipage}{.48\textwidth}
  \vspace{-2cm}
  \parbox[b]{.4\textwidth}{%
    Side \thepage\ af \pageref{LastPage}}\hfill
  \parbox[b]{.6\textwidth}{%
  \raggedleft \titlename}
  \rule{\textwidth}{1.5pt}\\
  \parbox[b]{.7\textwidth}{%
      \name }\hfill
  \parbox[b]{.3\textwidth}{%
  \raggedleft \email } 
  \end{minipage}\hspace{.02\textwidth}%
  \begin{minipage}{.5\textwidth}
  \vspace{-2cm}
  \includegraphics[width=\linewidth,height=70pt,keepaspectratio]{footer_logo}
  \end{minipage}%
  }
}

\newcommand{\namesigdate}[2][10cm]{%
  \begin{tabular}{@{}p{#1}@{}}
    #2 \\[2\normalbaselineskip] \hrule \\[0pt]
    {\small \textit{Signature}} \\[2\normalbaselineskip] \hrule \\[0pt]
    {\small \textit{Date}}
  \end{tabular}
}

\newcommand{\titlename}{Programmer som Data}
\newcommand{\name}{Jacob Benjamin Cholewa}
\newcommand{\email}{jbec@itu.dk}


\makeatletter
\NoBgThispage
 \title{\Huge \textbf{\titlename} \\ \huge Eksamens opgave Januar 2015}
 % Author
\author{\textsc{\name} \\ \textsc{\email}}
\begin{document}
\maketitle

\vspace*{\fill}
\begin{center}
\begin{large}
Jeg erklærer hermed at jeg selv har lavet hele denne eksamensbevarelse uden hjælp fra andre
\end{large}

\vspace*{2cm}
\namesigdate{}
\end{center}

\vspace*{\fill}


\newpage

\chapter*{Opgave 1}

I denne opgave vil jeg implementere en \textit{for-to-do} løkke

\begin{lstlisting}[language=C, label={lst:fortodo}]
for x = e1 to e2 do
  statement
\end{lstlisting}

der har samme opførsel som følgende kode stykke

\begin{lstlisting}[language=C, label={lst:fortodowhile}]
int x;
x = e1;
while(x <= e2){
  statement
  x = x+1;
}
\end{lstlisting}

Jeg viser i de næste to punkter hvilken kode jeg har ændret ved implementeringen og hvordan jeg har testet min løsning

\section*{Implementering}
I denne opgave har jeg lavet følgende ændringer til \texttt{CLex.fsl}. Ændringerne tilføjer de tre tokens \textit{for, to} og \textit{do} til lexer specifikationen. Dette gør jeg ved i parser specifikationen at oversætte \textit{for-to-do} løkken som vist i figur \ref{lst:fortodo} til en while løkke, som vist i figur \ref{lst:fortodowhile}, da \textit{while} løkken allerede er implementeret i microC syntaks og kompiler.

\begin{lstlisting}
let keyword s =
    match s with
    ...        
    | "for"     -> FOR 
    | "to"      -> TO 
    | "do"      -> DO 
    ...
;;
\end{lstlisting}

og følgende ændringer til \texttt{CPar.fsy}. Jeg har valgt at lave en løsning som implementere løkken ved hjælp af en while løkke og har derfor ikke ændret i microC syntaksen eller kompiler.

\begin{lstlisting}
...
%token CHAR ELSE IF INT NULL PRINT PRINTLN RETURN VOID WHILE FOR TO DO
...
StmtM:  /* No unbalanced if-else */
   ...
  | FOR NAME ASSIGN Expr TO Expr DO StmtM  { 
    Block [ 
      Dec(TypI, $2); 
      Stmt Expr(Assign(AccVar($2),$4)); 
      Stmt While(
        Prim2("<=", Access(AccVar($2)), $6),
        Block [
          Stmt $8; 
          Stmt Expr(Assign(AccVar($2), Prim2("+", Access(AccVar($2)), CstI 1)))
        ]
      )
    ] 
  }
;

StmtU:
  ...
  | FOR NAME ASSIGN Expr TO Expr DO StmtU  { 
    ... (Same as StmtM above) ...
  }
;
...
\end{lstlisting}


\section*{Evaluering}

Jeg har skrevet et test microC program til at teste min \textit{for-to-do} løkke implementation. 

\lstinputlisting[language=C]{MicroC/exfor.c}

der ved kørsel udprinter tal serien $0..n$ til stdOut. Det har jeg testet ved at køre 

\begin{lstlisting}[language=bash]
$ ./run exfor 5
0 1 2 3 4 5
Ran 0.0 seconds
\end{lstlisting}

og som man kan se fra output printer dette tal serien 0..5 som tilstræbet. Jeg bruger to hjælpe filer; henholdsvis run og Makefile

\subsubsection*{run}
\lstinputlisting[language=bash]{MicroC/run}

\subsubsection*{Makefile}
\lstinputlisting[language=make]{MicroC/Makefile}

\chapter*{Opgave 2}
I denne opgave har jeg implementeret \texttt{+=} og \texttt{-=} operatorer til MicroC sproget. Jeg vil i de næste to punkter vise hvordan jeg har lavet implementationen og hvordan jeg efterfølgende har testet den.

\section*{Implementering}

For at lave denne implementation har jeg ændret i \texttt{CLex.fsl, CPar.fsy, Absyn.fs} og \texttt{Comp.fs}.

Først har jeg tilføjet to nye tokens til lexer specifikationen.

\begin{lstlisting}[language=ML]
rule Token = parser
  ....
  | "+="            { EQPLUS }
  | "-="            { EQMINUS }
  ...
;;
\end{lstlisting}

Derefter har jeg tilføjet de nye tokens til parser specifikationen

\begin{lstlisting}[language=ML]
...
%token PLUS MINUS TIMES DIV MOD EQPLUS EQMINUS
...
ExprNotAccess:
  ...
  | Access EQPLUS Expr                  { Eqplus($1, $3)      }
  | Access EQMINUS Expr                 { Eqminus($1, $3)     }
;
\end{lstlisting}

Som det ses i parser specifikationen bruger jeg to nye kommandoer: \texttt{Eqplus} og \texttt{Eqminus} som nu skal tilføjes til den abstrakte syntaks.

\begin{lstlisting}[language=ML]
and expr =                                                         
  ...
  | Eqplus of access * expr          (* Equals plus operator        *)
  | Eqminus of access * expr         (* Equals minus operator       *)
  ...
\end{lstlisting}

Til sidst skal der nu tilføjes logik i kompileren til at forstå og udfører de nye operatorer. Det er i implementationen vigtigt at vestre siden af et udtryk, eg $a[0] += b$, kun evalueres en gang. Derfor starter jeg i kompileren med at hente adressen på variablen hvorefter den duplikeres med \texttt{DUP} kaldet da adressen skal bruges to gange til første at hente og derefter gemme variablen. Derefter bruger jeg \texttt{LDI} til at hente indholdet af variablen ind på stakken. Derefter evalueres den højre side af udtrykket og adderes så sammen med \texttt{ADD} før den nye værdi gemmes i variablen med \texttt{STI}.

\begin{lstlisting}[language=ML]
and cExpr (e : expr) (varEnv : varEnv) (funEnv : funEnv) : instr list = 
    ...
    | Eqplus (acc, e) ->  
                    cAccess acc varEnv funEnv 
                    @ [DUP;LDI] 
                    @ cExpr e varEnv funEnv 
                    @ [ADD; STI]  
    | Eqminus (acc, e) -> 
                    cAccess acc varEnv funEnv 
                    @ [DUP;LDI] 
                    @ cExpr e varEnv funEnv 
                    @ [SUB; STI] 
\end{lstlisting}


section*{Evaluering}

Jeg har skrevet et test microC program til at teste operatorerne.

\lstinputlisting[language=C]{MicroC/exeqplus.c}

der ved kørsel udprinter $100 1$ til stdOut. Det er også det ventede resultat da $0+100$ giver $100$ som bliver udskrevet og da $i$ kun er blevet inkrementeret en gang.  Det har jeg testet ved at køre 

\begin{lstlisting}[language=bash]
$ ./run exeqplus
100 1
Ran 0.0 seconds
\end{lstlisting}

Jeg har brugt samme hjælpe metoder som i opgave 1

\chapter*{Opgave 3}
I denne opgave har jeg implementeret kommandoen \texttt{INDEX} til den abstrakte maskine. Denne kommando skal checke, givet et array og et indeks, om arrayet indeholder indekset og i givet fald så hente adressen til elementet på indekset.

\section*{Implementering}
Jeg har i denne opgave ændret i tre filer: \texttt{Machine.fs, Machine.java} og \texttt{Comp.fs}

Jeg starede med at tilføje den nye kommando til \texttt{Machine.fs}

\begin{lstlisting}[language=ML]

type instr =
  ...
  | STOP                               (* halt the abstract machine       *)
  | INDEX                              (* get s[s[sp-1]] + s[sp]          *)
...
let CODESTOP   = 25
let CODEINDEX  = 26;
...
let makelabenv (addr, labenv) instr = 
    match instr with
    ...
    | STOP           -> (addr+1, labenv)
    | INDEX          -> (addr+1, labenv)

let rec emitints getlab instr ints = 
    match instr with
    | Label lab      -> ints
    ...
    | STOP           -> CODESTOP   :: ints
    | INDEX          -> CODEINDEX  :: ints
\end{lstlisting}

Nu da vi har tilføjet til dem vores abstrakte maskines sytaks skal vi nu ændre den abstrakte maskine til at kunne udfører kommandoen.

\begin{lstlisting}[language=java]
final static int 
  ...
  STOP = 25,
  INDEX = 26;
...

static int execcode(int[] p, int[] s, int[] iargs, boolean trace) {
  int bp = -999;  // Base pointer, for local variable access 
  int sp = -1;  // Stack top pointer
  int pc = 0;   // Program counter: next instruction
  for (;;) {
    if (trace) printsppc(s, bp, sp, p, pc);
    switch (p[pc++]) {
      ...
      case INDEX:
        int a,q,i;
        a = s[sp-1]; q = s[a]; i = s[sp];
        if(0 <= i && i < (a-q)) s[--sp] = q+i;
        else throw new ArrayIndexOutOfBoundsException(i);
        break;
      ...
    }
  }
}
\end{lstlisting}

Når kommandoen udførers ligger der adressen til arrayet og indekset på stakken. Vi henter derfor først $a$, som er adressen på arrayet, $q$, som er adressen på det første element i arrayet, og $i$, som er indekset på det element vi vil hente. Da længden af arrayet kan udregnes ved $n = a - q$ checker vi først om indekset er inden for arrayets længde. Hvis indekset er inden for arrayets længde lægger vi elementets adresse på stakken, $addr = q+i$, ellers kastes en exception.

Til sidst skal vi ændre kompileren til bruge kommandoen \texttt{INDEX}.

\begin{lstlisting}[language=ML]
and cAccess access varEnv funEnv : instr list =
    match access with 
    ...
    | AccIndex(acc, idx) -> cAccess acc varEnv funEnv 
                             @ cExpr idx varEnv funEnv @ [INDEX]
\end{lstlisting}

\section*{Evaluering}

For at teste implementationen har jeg skrevet et lille microC program.

\lstinputlisting[language=C]{MicroC/exindex.c}

Dette program tilgår et array, og derfor vil \texttt{INDEX} komandoen blive kaldt. For både at teste gyldige og ugyldige array tilgange har jeg lavet tre kørsler

\begin{lstlisting}[language=bash]
$ ./run exindex 1 1
20 62
Ran 0.0 seconds
\end{lstlisting}

\begin{lstlisting}[language=bash]
$ ./run exindex 1 3
20 Exception in thread "main" java.lang.ArrayIndexOutOfBoundsException: Array index out of range: 3
  at Machine.execcode(Machine.java:141)
  at Machine.execute(Machine.java:53)
  at Machine.main(Machine.java:24)
\end{lstlisting}



\chapter*{Opgave 4}

\begin{lstlisting}[language=ML]
let rec tExpr (e:expr) (varEnv:varEnv) (funEnv:funEnv) : typ = 
        match e with
        | Access acc     -> tAccess acc varEnv funEnv
        | Assign(acc, e) -> tAccess acc varEnv funEnv
        | CstI i         -> TypI
        | Addr acc       -> tAccess acc varEnv funEnv
        | _              -> failwith "Not implemented"
and tAccess (access:access) (varEnv:varEnv) (funEnv:funEnv) : typ =
        match access with
        | AccVar s -> 
                match lookup (fst varEnv) s with
                | _, t -> t
        | AccDeref e -> 
                match (tExpr e varEnv funEnv) with
                | TypI    -> TypP (TypI)
                | TypP(t) -> TypP(TypP(t))
                | _       -> failwith "Not type correct"
        | AccIndex(a,e) when tExpr e varEnv funEnv = TypI -> 
                TypA (tAccess a varEnv funEnv, None)
        | _ ->  failwith "Not type correct"
\end{lstlisting}

\label{LastPage}
\end{document}